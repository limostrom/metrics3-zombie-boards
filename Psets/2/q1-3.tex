% Options for packages loaded elsewhere
\PassOptionsToPackage{unicode}{hyperref}
\PassOptionsToPackage{hyphens}{url}
%
\documentclass[
]{article}
\usepackage{amsmath,amssymb}
\usepackage{lmodern}
\usepackage{iftex}
\ifPDFTeX
  \usepackage[T1]{fontenc}
  \usepackage[utf8]{inputenc}
  \usepackage{textcomp} % provide euro and other symbols
\else % if luatex or xetex
  \usepackage{unicode-math}
  \defaultfontfeatures{Scale=MatchLowercase}
  \defaultfontfeatures[\rmfamily]{Ligatures=TeX,Scale=1}
\fi
% Use upquote if available, for straight quotes in verbatim environments
\IfFileExists{upquote.sty}{\usepackage{upquote}}{}
\IfFileExists{microtype.sty}{% use microtype if available
  \usepackage[]{microtype}
  \UseMicrotypeSet[protrusion]{basicmath} % disable protrusion for tt fonts
}{}
\makeatletter
\@ifundefined{KOMAClassName}{% if non-KOMA class
  \IfFileExists{parskip.sty}{%
    \usepackage{parskip}
  }{% else
    \setlength{\parindent}{0pt}
    \setlength{\parskip}{6pt plus 2pt minus 1pt}}
}{% if KOMA class
  \KOMAoptions{parskip=half}}
\makeatother
\usepackage{xcolor}
\usepackage[margin=1in]{geometry}
\usepackage{color}
\usepackage{fancyvrb}
\newcommand{\VerbBar}{|}
\newcommand{\VERB}{\Verb[commandchars=\\\{\}]}
\DefineVerbatimEnvironment{Highlighting}{Verbatim}{commandchars=\\\{\}}
% Add ',fontsize=\small' for more characters per line
\usepackage{framed}
\definecolor{shadecolor}{RGB}{248,248,248}
\newenvironment{Shaded}{\begin{snugshade}}{\end{snugshade}}
\newcommand{\AlertTok}[1]{\textcolor[rgb]{0.94,0.16,0.16}{#1}}
\newcommand{\AnnotationTok}[1]{\textcolor[rgb]{0.56,0.35,0.01}{\textbf{\textit{#1}}}}
\newcommand{\AttributeTok}[1]{\textcolor[rgb]{0.77,0.63,0.00}{#1}}
\newcommand{\BaseNTok}[1]{\textcolor[rgb]{0.00,0.00,0.81}{#1}}
\newcommand{\BuiltInTok}[1]{#1}
\newcommand{\CharTok}[1]{\textcolor[rgb]{0.31,0.60,0.02}{#1}}
\newcommand{\CommentTok}[1]{\textcolor[rgb]{0.56,0.35,0.01}{\textit{#1}}}
\newcommand{\CommentVarTok}[1]{\textcolor[rgb]{0.56,0.35,0.01}{\textbf{\textit{#1}}}}
\newcommand{\ConstantTok}[1]{\textcolor[rgb]{0.00,0.00,0.00}{#1}}
\newcommand{\ControlFlowTok}[1]{\textcolor[rgb]{0.13,0.29,0.53}{\textbf{#1}}}
\newcommand{\DataTypeTok}[1]{\textcolor[rgb]{0.13,0.29,0.53}{#1}}
\newcommand{\DecValTok}[1]{\textcolor[rgb]{0.00,0.00,0.81}{#1}}
\newcommand{\DocumentationTok}[1]{\textcolor[rgb]{0.56,0.35,0.01}{\textbf{\textit{#1}}}}
\newcommand{\ErrorTok}[1]{\textcolor[rgb]{0.64,0.00,0.00}{\textbf{#1}}}
\newcommand{\ExtensionTok}[1]{#1}
\newcommand{\FloatTok}[1]{\textcolor[rgb]{0.00,0.00,0.81}{#1}}
\newcommand{\FunctionTok}[1]{\textcolor[rgb]{0.00,0.00,0.00}{#1}}
\newcommand{\ImportTok}[1]{#1}
\newcommand{\InformationTok}[1]{\textcolor[rgb]{0.56,0.35,0.01}{\textbf{\textit{#1}}}}
\newcommand{\KeywordTok}[1]{\textcolor[rgb]{0.13,0.29,0.53}{\textbf{#1}}}
\newcommand{\NormalTok}[1]{#1}
\newcommand{\OperatorTok}[1]{\textcolor[rgb]{0.81,0.36,0.00}{\textbf{#1}}}
\newcommand{\OtherTok}[1]{\textcolor[rgb]{0.56,0.35,0.01}{#1}}
\newcommand{\PreprocessorTok}[1]{\textcolor[rgb]{0.56,0.35,0.01}{\textit{#1}}}
\newcommand{\RegionMarkerTok}[1]{#1}
\newcommand{\SpecialCharTok}[1]{\textcolor[rgb]{0.00,0.00,0.00}{#1}}
\newcommand{\SpecialStringTok}[1]{\textcolor[rgb]{0.31,0.60,0.02}{#1}}
\newcommand{\StringTok}[1]{\textcolor[rgb]{0.31,0.60,0.02}{#1}}
\newcommand{\VariableTok}[1]{\textcolor[rgb]{0.00,0.00,0.00}{#1}}
\newcommand{\VerbatimStringTok}[1]{\textcolor[rgb]{0.31,0.60,0.02}{#1}}
\newcommand{\WarningTok}[1]{\textcolor[rgb]{0.56,0.35,0.01}{\textbf{\textit{#1}}}}
\usepackage{graphicx}
\makeatletter
\def\maxwidth{\ifdim\Gin@nat@width>\linewidth\linewidth\else\Gin@nat@width\fi}
\def\maxheight{\ifdim\Gin@nat@height>\textheight\textheight\else\Gin@nat@height\fi}
\makeatother
% Scale images if necessary, so that they will not overflow the page
% margins by default, and it is still possible to overwrite the defaults
% using explicit options in \includegraphics[width, height, ...]{}
\setkeys{Gin}{width=\maxwidth,height=\maxheight,keepaspectratio}
% Set default figure placement to htbp
\makeatletter
\def\fps@figure{htbp}
\makeatother
\setlength{\emergencystretch}{3em} % prevent overfull lines
\providecommand{\tightlist}{%
  \setlength{\itemsep}{0pt}\setlength{\parskip}{0pt}}
\setcounter{secnumdepth}{-\maxdimen} % remove section numbering
\usepackage{flafter}
\usepackage{booktabs}
\usepackage{longtable}
\usepackage{array}
\usepackage{multirow}
\usepackage{wrapfig}
\usepackage{float}
\usepackage{colortbl}
\usepackage{pdflscape}
\usepackage{tabu}
\usepackage{threeparttable}
\usepackage{threeparttablex}
\usepackage[normalem]{ulem}
\usepackage{makecell}
\usepackage{xcolor}
\ifLuaTeX
  \usepackage{selnolig}  % disable illegal ligatures
\fi
\IfFileExists{bookmark.sty}{\usepackage{bookmark}}{\usepackage{hyperref}}
\IfFileExists{xurl.sty}{\usepackage{xurl}}{} % add URL line breaks if available
\urlstyle{same} % disable monospaced font for URLs
\hypersetup{
  pdftitle={41903 Pset 2},
  pdfauthor={Andrew McKinley, Lauren Mostrom, Pietro Ramella, Francisco Ruela, and Bohan Yang},
  hidelinks,
  pdfcreator={LaTeX via pandoc}}

\title{41903 Pset 2}
\author{Andrew McKinley, Lauren Mostrom, Pietro Ramella, Francisco
Ruela, and Bohan Yang}
\date{May 01, 2022}

\begin{document}
\maketitle

\hypertarget{question-1}{%
\subsection{Question 1}\label{question-1}}

\textit{Carefully evaluate the following comment. "I'm interested in estimating a model for the quantity of demand for widgets as a function of price, and I have some variables that tell me about the costs of producing widgets that I know are unrelated to demand shocks. I've estimated my model using OLS and 2SLS and have found that OLS produces better within sample forecasts and that the OLS forecasts are also better within a holdout sample I set aside before estimating the model. I also know that if a model doesn't forecast well, it's not a good model; so I'm going to use the OLS estimates to gauge the possible effects on quantity sold of implementing a $10 \%$ increase in price."}
\vspace{2mm} First and foremost we know that trying to run traditional
OLS models to estimate demand run into simultaneity. In fact this is
quite literally the motivating example used throughout Chapter 3 of
Hayashi. If we describe the economy as a system of equations we
immediately notice that we have two \textit{simultaneous} equations as a
function of price (noting that in market equilibrium
\(q_i^d = q_i^s = q_i)\): \begin{align*}
\tag{demand} q_i &=\alpha_0 + \alpha_1p_1 + u_i \\
\tag{supply} q_i &=\beta_0 + \beta_1p_1 + v_i
\end{align*} If you are interested in \(q_i\) or \(p_i\) and solve the
above equations to isolate the variable, you quickly see that they are
functions of both errors and in OLS we will have a biased estimate.

The motivation for 2SLS is that we have a consistent estimator, even in
cases of endogeniety. With a valid, strong instrument, as suggested by
the statement, 2SLS will be the more ``correct'' model. If the
instrument is weak, then we can make an argument that in sum cases OLS
is the more precise estimate (though this still concedes a falsehood in
the statement)

Finally, forecasting isn't the end all be all of good models. At the end
of the day if a variable is random and has high variance a perfect the
TRUTH will be unforecastable, so even if the model is the ``truth'' we
will have bad forecastability, but a complete understanding of the
underlying mechanisms of the event/phenomenon we are interested.

\newpage

\hypertarget{question-2}{%
\subsection{Question 2}\label{question-2}}

\hypertarget{a}{%
\subsubsection{(a)}\label{a}}

We consider the following model:

\[ \log(bwght) = \beta_0 + \beta_1 male + \beta_2 parity + \beta_3 \log(faminc) + \beta_4 packs + u \]

We might expect \(packs\) to be correlated with \(u\) for a number of
reasons. One example could be age cohort effects: older mothers are more
likely to give birth to underweight babies, and may have grown into
adulthood at a time when the dangers of smoking were not as well
understood or communicated in schools. Another reason may be underlying
health conditions of the mother, such as stress and anxiety, eating
disorders, and conditions that cause chronic pain that could induce the
mother to smoke more as a self-medicating mechanism and would reduce the
birthweight of the baby. This could result in omitted variable bias,
where \(\beta_4\) would overstate the effect of smoking on baby
birthweight because it would pick up the effects of the underlying
health condition(s) correlated both with birthweight and smoking.

\hypertarget{b}{%
\subsubsection{(b)}\label{b}}

For cigarette prices to be a suitable instrument we need for it to be
significantly correlated with \(packs\) (relevance) but not with \(u\)
(exclusion). We would expect the relevance condition to hold, but it may
be weak. From basic laws of demand we would expect high cigarette prices
to reduce cigarette consumption, but because of the addictive qualities
of nicotine we may also worry that demand for cigarettes is relatively
inelastic. The exclusion restriction is also somewhat dubious because
state and local policies that raise the prices of cigarettes may also be
correlated with the quality of healthcare access in those states, which
could be correlated with women's consumption of cigarettes to self-treat
underlying health conditions, as discussed in (a).

\hypertarget{c}{%
\subsubsection{(c)}\label{c}}

\begin{Shaded}
\begin{Highlighting}[]
\CommentTok{\# Load data BWGHT.raw into R}
\CommentTok{\#setwd("C:/Users/17036/OneDrive/Documents/GitHub/metrics3{-}zombie{-}boards/Psets/2/PS2Data")}
\NormalTok{BWGHT }\OtherTok{\textless{}{-}} \FunctionTok{read.delim}\NormalTok{(}\StringTok{"PS2Data/BWGHT\_RAW.txt"}\NormalTok{, }\AttributeTok{header=}\ConstantTok{FALSE}\NormalTok{)}
\FunctionTok{colnames}\NormalTok{(BWGHT) }\OtherTok{\textless{}{-}} \FunctionTok{c}\NormalTok{(}\StringTok{"faminc"}\NormalTok{,}\StringTok{"cigtax"}\NormalTok{,}\StringTok{"cigprice"}\NormalTok{,}\StringTok{"bwght"}\NormalTok{,}\StringTok{"fatheduc"}\NormalTok{,}\StringTok{"motheduc"}\NormalTok{,}\StringTok{"parity"}\NormalTok{,}
                     \StringTok{"male"}\NormalTok{,}\StringTok{"white"}\NormalTok{,}\StringTok{"cigs"}\NormalTok{,}\StringTok{"lbwght"}\NormalTok{,}\StringTok{"bwghtlbs"}\NormalTok{,}\StringTok{"packs"}\NormalTok{,}\StringTok{"lfaminc"}\NormalTok{)  }


\CommentTok{\# select variables}
\NormalTok{lbwght }\OtherTok{\textless{}{-}}\NormalTok{ BWGHT}\SpecialCharTok{$}\NormalTok{lbwght }\CommentTok{\# dependent variable: log(bwght)}
\NormalTok{male }\OtherTok{\textless{}{-}}\NormalTok{ BWGHT}\SpecialCharTok{$}\NormalTok{male }\CommentTok{\# independent variable: male}
\NormalTok{parity }\OtherTok{\textless{}{-}}\NormalTok{ BWGHT}\SpecialCharTok{$}\NormalTok{parity }\CommentTok{\# independent variable: parity}
\NormalTok{lfaminc }\OtherTok{\textless{}{-}}\NormalTok{ BWGHT}\SpecialCharTok{$}\NormalTok{lfaminc }\CommentTok{\# independent variable: log(faminc)}
\NormalTok{packs }\OtherTok{\textless{}{-}}\NormalTok{ BWGHT}\SpecialCharTok{$}\NormalTok{packs }\CommentTok{\# independent variable: packs}

\CommentTok{\# (i) OLS }
\NormalTok{reg\_ols }\OtherTok{\textless{}{-}} \FunctionTok{lm}\NormalTok{(lbwght }\SpecialCharTok{\textasciitilde{}}\NormalTok{ male }\SpecialCharTok{+}\NormalTok{ parity }\SpecialCharTok{+}\NormalTok{ lfaminc }\SpecialCharTok{+}\NormalTok{ packs)}
\NormalTok{coef\_ols }\OtherTok{\textless{}{-}} \FunctionTok{coef}\NormalTok{(reg\_ols)}
\NormalTok{HCV.coef\_ols }\OtherTok{\textless{}{-}} \FunctionTok{vcovHC}\NormalTok{(reg\_ols, }\AttributeTok{type =} \StringTok{\textquotesingle{}HC\textquotesingle{}}\NormalTok{)}
\NormalTok{coef\_ols.se }\OtherTok{\textless{}{-}} \FunctionTok{sqrt}\NormalTok{(}\FunctionTok{diag}\NormalTok{(HCV.coef\_ols)) }\CommentTok{\# White standard errors}
\NormalTok{output\_ols }\OtherTok{\textless{}{-}} \FunctionTok{cbind}\NormalTok{(coef\_ols,coef\_ols.se)}

\CommentTok{\# (ii) 2SLS}
\NormalTok{cigprice }\OtherTok{\textless{}{-}}\NormalTok{ BWGHT}\SpecialCharTok{$}\NormalTok{cigprice }\CommentTok{\# IV: cigprice}
\CommentTok{\# Note that 2SLS should be estimated in one step to get correct standard errors}
\NormalTok{reg\_2sls }\OtherTok{\textless{}{-}} \FunctionTok{ivreg}\NormalTok{(lbwght }\SpecialCharTok{\textasciitilde{}}\NormalTok{ packs}\SpecialCharTok{+}\NormalTok{male}\SpecialCharTok{+}\NormalTok{parity}\SpecialCharTok{+}\NormalTok{lfaminc}\SpecialCharTok{|}\NormalTok{cigprice}\SpecialCharTok{+}\NormalTok{male}\SpecialCharTok{+}\NormalTok{parity}\SpecialCharTok{+}\NormalTok{lfaminc)}
\NormalTok{coef\_2sls }\OtherTok{\textless{}{-}} \FunctionTok{coef}\NormalTok{(reg\_2sls)}
\NormalTok{HCV.coef\_2sls }\OtherTok{\textless{}{-}} \FunctionTok{vcovHC}\NormalTok{(reg\_2sls, }\AttributeTok{type =} \StringTok{\textquotesingle{}HC\textquotesingle{}}\NormalTok{)}
\NormalTok{coef\_2sls.se }\OtherTok{\textless{}{-}} \FunctionTok{sqrt}\NormalTok{(}\FunctionTok{diag}\NormalTok{(HCV.coef\_2sls)) }\CommentTok{\# White standard errors}
\NormalTok{output\_2sls }\OtherTok{\textless{}{-}} \FunctionTok{cbind}\NormalTok{(coef\_2sls,coef\_2sls.se)}
\end{Highlighting}
\end{Shaded}

The results shown in Table 1 show that the OLS estimate for the
coefficient on \(packs\) is negative and statistically significant,
which is consistent with what we expect about smoking reducing babies'
birthweights. However the 2SLS estimate is very imprecise (SE=1.0863),
and the point estimate is large (0.7971) and positive. The
interpretation of this would be that an additional pack of cigarettes
consumed by the mother \emph{increases} the birthweight of her baby by
80\%, which simply cannot be true. The discrepancy between these results
suggests we should reconsider whether \(cigprice\) is a valid instrument
for \(packs\).

\hypertarget{d}{%
\subsubsection{(d)}\label{d}}

\begin{Shaded}
\begin{Highlighting}[]
\CommentTok{\# First stage: regress X (endogenous variable) on Z (instrumental variable)}
\NormalTok{stage1 }\OtherTok{\textless{}{-}} \FunctionTok{lm}\NormalTok{(packs }\SpecialCharTok{\textasciitilde{}}\NormalTok{ male }\SpecialCharTok{+}\NormalTok{ parity }\SpecialCharTok{+}\NormalTok{ lfaminc }\SpecialCharTok{+}\NormalTok{ cigprice)}
\CommentTok{\# predetermined regressors should be included in the list of instruments}
\NormalTok{coef\_stage1 }\OtherTok{\textless{}{-}} \FunctionTok{coef}\NormalTok{(stage1)}
\NormalTok{HCV.coef\_stage1 }\OtherTok{\textless{}{-}} \FunctionTok{vcovHC}\NormalTok{(stage1, }\AttributeTok{type =} \StringTok{\textquotesingle{}HC\textquotesingle{}}\NormalTok{)}
\NormalTok{coef\_stage1.se }\OtherTok{\textless{}{-}} \FunctionTok{sqrt}\NormalTok{(}\FunctionTok{diag}\NormalTok{(HCV.coef\_stage1)) }\CommentTok{\# White standard errors}
\NormalTok{output\_stage1 }\OtherTok{\textless{}{-}} \FunctionTok{cbind}\NormalTok{(coef\_stage1,coef\_stage1.se)}
\end{Highlighting}
\end{Shaded}

We estimate the reduced form for \(packs\) as follows:
\[ \hat{packs} = \gamma_0 + \gamma_1 male + \gamma_2 parity + \gamma_3 \log(faminc) +\gamma_4 cigprice \]

As shown in Table 2, the coefficient on \(cigprice\) is very small and
not at all significant. For \(cigprice\) to be a valid instrument for
\(packs\), even if we were able to tell a good story about why
\(cigprice\) should be uncorrelated with \(u\) (which is of course
untestable), we would still need to be confident that the matrix \(ZX'\)
is very far from zero. However since the estimate for \(\gamma_4\) is
very close to zero and nowhere near being statistically significant even
at the 5\% level, \(cigprice\) does not satisfy the relevance condition
and cannot be taken as a valid instrument.

The results in Table 2 help explain why in part (c) the 2SLS results
were so far off from the OLS results. This is because the instrument and
covariates are very poor predictors of \(packs\), so it makes sense that
the point estimate on packs from 2SLS was so imprecise that it was
basically indistinguishable from zero (despite the point estimate itself
being large and positive). This is also a good example of a time when a
point estimate is much less informative (and, in fact, misleading), and
reporting a confidence interval would be much more useful to the reader.

\begin{Shaded}
\begin{Highlighting}[]
\CommentTok{\# OLS vs 2SLS  Table}
\FunctionTok{texreg}\NormalTok{(}\FunctionTok{list}\NormalTok{(reg\_ols, reg\_2sls), }\AttributeTok{digits=}\DecValTok{4}\NormalTok{, }\AttributeTok{caption.above=}\ConstantTok{TRUE}\NormalTok{)}
\end{Highlighting}
\end{Shaded}

\begin{table}
\caption{Statistical models}
\begin{center}
\begin{tabular}{l c c}
\hline
 & Model 1 & Model 2 \\
\hline
(Intercept) & $4.6756^{***}$  & $4.4679^{***}$ \\
            & $(0.0219)$      & $(0.2588)$     \\
male        & $0.0262^{**}$   & $0.0298$       \\
            & $(0.0101)$      & $(0.0178)$     \\
parity      & $0.0147^{**}$   & $-0.0012$      \\
            & $(0.0057)$      & $(0.0219)$     \\
lfaminc     & $0.0180^{**}$   & $0.0636$       \\
            & $(0.0056)$      & $(0.0570)$     \\
packs       & $-0.0837^{***}$ & $0.7971$       \\
            & $(0.0171)$      & $(1.0863)$     \\
\hline
R$^2$       & $0.0350$        & $-1.8118$      \\
Adj. R$^2$  & $0.0322$        & $-1.8199$      \\
Num. obs.   & $1388$          & $1388$         \\
\hline
\multicolumn{3}{l}{\scriptsize{$^{***}p<0.001$; $^{**}p<0.01$; $^{*}p<0.05$}}
\end{tabular}
\label{table:coefficients}
\end{center}
\end{table}

\begin{Shaded}
\begin{Highlighting}[]
\CommentTok{\# Reduced Form Table}
\FunctionTok{texreg}\NormalTok{(stage1, }\AttributeTok{digits=}\DecValTok{4}\NormalTok{, }\AttributeTok{caption.above=}\ConstantTok{TRUE}\NormalTok{)}
\end{Highlighting}
\end{Shaded}

\begin{table}
\caption{Statistical models}
\begin{center}
\begin{tabular}{l c}
\hline
 & Model 1 \\
\hline
(Intercept) & $0.1374$        \\
            & $(0.1040)$      \\
male        & $-0.0047$       \\
            & $(0.0159)$      \\
parity      & $0.0181^{*}$    \\
            & $(0.0089)$      \\
lfaminc     & $-0.0526^{***}$ \\
            & $(0.0087)$      \\
cigprice    & $0.0008$        \\
            & $(0.0008)$      \\
\hline
R$^2$       & $0.0305$        \\
Adj. R$^2$  & $0.0276$        \\
Num. obs.   & $1388$          \\
\hline
\multicolumn{2}{l}{\scriptsize{$^{***}p<0.001$; $^{**}p<0.01$; $^{*}p<0.05$}}
\end{tabular}
\label{table:coefficients}
\end{center}
\end{table}

\newpage

\hypertarget{question-3}{%
\subsection{Question 3}\label{question-3}}

\begin{Shaded}
\begin{Highlighting}[]
\CommentTok{\# Prepare the data}
\NormalTok{df }\OtherTok{\textless{}{-}} \FunctionTok{read.table}\NormalTok{(}\StringTok{"PS2Data/CARD.raw"}\NormalTok{, }\AttributeTok{quote=}\StringTok{"}\SpecialCharTok{\textbackslash{}"}\StringTok{"}\NormalTok{, }\AttributeTok{comment.char=}\StringTok{""}\NormalTok{)}

\NormalTok{name\_string }\OtherTok{\textless{}{-}} \StringTok{"id   nearc2    nearc4    educ      age       fatheduc  motheduc  }
\StringTok{weight   momdad14  sinmom14  step14    reg661    reg662    reg663    reg664   }
\StringTok{reg665 reg666    reg667    reg668    reg669    south66   black     smsa   south    }
\StringTok{smsa66    wage      enroll    KWW       IQ        married   libcrd14  exper    }
\StringTok{lwage     expersq   "}

\NormalTok{name\_string }\OtherTok{\textless{}{-}} \FunctionTok{gsub}\NormalTok{(}\StringTok{"[}\SpecialCharTok{\textbackslash{}r\textbackslash{}n}\StringTok{]"}\NormalTok{, }\StringTok{" "}\NormalTok{, name\_string)}
\NormalTok{name\_string }\OtherTok{\textless{}{-}} \FunctionTok{strsplit}\NormalTok{(name\_string, }\StringTok{" "}\NormalTok{)}

\NormalTok{name\_vec }\OtherTok{\textless{}{-}} \FunctionTok{vector}\NormalTok{()}
\ControlFlowTok{for}\NormalTok{ (i }\ControlFlowTok{in}\NormalTok{ name\_string[[}\DecValTok{1}\NormalTok{]]) \{}
  \ControlFlowTok{if}\NormalTok{ (i }\SpecialCharTok{!=} \StringTok{"1"} \SpecialCharTok{\&}\NormalTok{ i }\SpecialCharTok{!=} \StringTok{""}\NormalTok{)\{}
\NormalTok{    name\_vec }\OtherTok{\textless{}{-}} \FunctionTok{append}\NormalTok{(name\_vec, i)}
\NormalTok{  \}}
\NormalTok{\}}
\FunctionTok{colnames}\NormalTok{(df) }\OtherTok{\textless{}{-}}\NormalTok{ name\_vec}
\end{Highlighting}
\end{Shaded}

\hypertarget{a-1}{%
\subsubsection{(a)}\label{a-1}}

In the table below, the \(iid\) and \(robust\) standard errors are quite
similar (\(robust\) standard errors are usually a little larger), so the
inference (ttest and pvalue) is also quite similar.

The \(iid\) standard error assumes that the variance of the error term
is constant and does not depend on independent variables. However, the
\(robust\) standard error does not assume this, so it can work under
both homoskedasticity and heteroskedasticity.

\begin{Shaded}
\begin{Highlighting}[]
\CommentTok{\# Homoskedastic}
\NormalTok{homo }\OtherTok{\textless{}{-}} \FunctionTok{feols}\NormalTok{(lwage }\SpecialCharTok{\textasciitilde{}}\NormalTok{ educ }\SpecialCharTok{+}\NormalTok{ exper }\SpecialCharTok{+}\NormalTok{ expersq }\SpecialCharTok{+}\NormalTok{ black }\SpecialCharTok{+}\NormalTok{ south }\SpecialCharTok{+}\NormalTok{ smsa }\SpecialCharTok{+}\NormalTok{ smsa66 }\SpecialCharTok{+}
\NormalTok{              reg661 }\SpecialCharTok{+}\NormalTok{ reg662 }\SpecialCharTok{+}\NormalTok{ reg663 }\SpecialCharTok{+}\NormalTok{ reg664 }\SpecialCharTok{+}\NormalTok{ reg665 }\SpecialCharTok{+}\NormalTok{ reg666 }\SpecialCharTok{+}\NormalTok{ reg667 }\SpecialCharTok{+}\NormalTok{ reg668,}
              \AttributeTok{se =} \StringTok{"iid"}\NormalTok{, df)}

\CommentTok{\# Heteroskedastic}
\NormalTok{hetero }\OtherTok{\textless{}{-}} \FunctionTok{feols}\NormalTok{(lwage }\SpecialCharTok{\textasciitilde{}}\NormalTok{ educ }\SpecialCharTok{+}\NormalTok{ exper }\SpecialCharTok{+}\NormalTok{ expersq }\SpecialCharTok{+}\NormalTok{ black }\SpecialCharTok{+}\NormalTok{ south }\SpecialCharTok{+}\NormalTok{ smsa }\SpecialCharTok{+}\NormalTok{ smsa66 }\SpecialCharTok{+}
\NormalTok{                reg661 }\SpecialCharTok{+}\NormalTok{ reg662 }\SpecialCharTok{+}\NormalTok{ reg663 }\SpecialCharTok{+}\NormalTok{ reg664 }\SpecialCharTok{+}\NormalTok{ reg665 }\SpecialCharTok{+}\NormalTok{ reg666 }\SpecialCharTok{+}\NormalTok{ reg667 }\SpecialCharTok{+}\NormalTok{ reg668,}
                \AttributeTok{se =} \StringTok{"hc1"}\NormalTok{, df)}

\NormalTok{out }\OtherTok{\textless{}{-}} \FunctionTok{etable}\NormalTok{(homo, hetero, }\AttributeTok{tex =} \ConstantTok{TRUE}\NormalTok{, }\AttributeTok{se.row =} \ConstantTok{TRUE}\NormalTok{) }
\end{Highlighting}
\end{Shaded}

\begin{Shaded}
\begin{Highlighting}[]
\NormalTok{knitr}\SpecialCharTok{::}\FunctionTok{asis\_output}\NormalTok{(}\FunctionTok{c}\NormalTok{(}\StringTok{"}\SpecialCharTok{\textbackslash{}\textbackslash{}}\StringTok{begin\{center\}"}\NormalTok{, out, }\StringTok{"}\SpecialCharTok{\textbackslash{}\textbackslash{}}\StringTok{end\{center\}"}\NormalTok{)) }
\end{Highlighting}
\end{Shaded}

\begin{center}\begingroup\centering\begin{tabular}{lcc}   \tabularnewline \midrule \midrule   Dependent Variable: & \multicolumn{2}{c}{lwage}\\   Model:          & (1)             & (2)\\     \midrule   \emph{Variables}\\   (Intercept)     & 4.739$^{***}$   & 4.739$^{***}$\\                      & (0.0715)        & (0.0746)\\      educ            & 0.0747$^{***}$  & 0.0747$^{***}$\\                      & (0.0035)        & (0.0036)\\      exper           & 0.0848$^{***}$  & 0.0848$^{***}$\\                      & (0.0066)        & (0.0068)\\      expersq         & -0.0023$^{***}$ & -0.0023$^{***}$\\                      & (0.0003)        & (0.0003)\\      black           & -0.1990$^{***}$ & -0.1990$^{***}$\\                      & (0.0182)        & (0.0182)\\      south           & -0.1480$^{***}$ & -0.1480$^{***}$\\                      & (0.0260)        & (0.0280)\\      smsa            & 0.1364$^{***}$  & 0.1364$^{***}$\\                      & (0.0201)        & (0.0192)\\      smsa66          & 0.0262          & 0.0262\\                      & (0.0194)        & (0.0186)\\      reg661          & -0.1186$^{***}$ & -0.1186$^{***}$\\                      & (0.0388)        & (0.0388)\\      reg662          & -0.0222         & -0.0222\\                      & (0.0283)        & (0.0299)\\      reg663          & 0.0260          & 0.0260\\                      & (0.0274)        & (0.0285)\\      reg664          & -0.0635$^{*}$   & -0.0635$^{*}$\\                      & (0.0357)        & (0.0368)\\      reg665          & 0.0095          & 0.0095\\                      & (0.0361)        & (0.0387)\\      reg666          & 0.0220          & 0.0220\\                      & (0.0401)        & (0.0411)\\      reg667          & -0.0006         & -0.0006\\                      & (0.0394)        & (0.0415)\\      reg668          & -0.1750$^{***}$ & -0.1750$^{***}$\\                      & (0.0463)        & (0.0470)\\      \midrule   \emph{Fit statistics}\\   Standard-Errors & IID             & Heteroskedasticity-robust \\      Observations    & 3,010           & 3,010\\     R$^2$           & 0.29984         & 0.29984\\     Adjusted R$^2$  & 0.29633         & 0.29633\\     \midrule \midrule   \multicolumn{3}{l}{\emph{Signif. Codes: ***: 0.01, **: 0.05, *: 0.1}}\\\end{tabular}\par\endgroup\end{center}

\newpage

\hypertarget{b-1}{%
\subsubsection{(b)}\label{b-1}}

There exists a practically and statistically significant partial
correlation between \(educ\) and \(nearc4\): the coefficient of
\(nearc4\) is \(0.32\), which means indivisuals near 4 yr college have
additional \(0.32\) years of schooling. It's significant under \(1\%\)
level with both \(iid\) and \(robust\) standard errors.

For some variables, the \(robust\) standard error is a little larger,
and for other variables, the \(robust\) standard error is a little
smaller. But they are still quite similar and the inference (ttest and
pvalue) gives same conclusions. For \(near4\), the \(robust\) standard
error is a little smaller, but the coefficient is both significant under
\(1\%\) level.

\begin{Shaded}
\begin{Highlighting}[]
\NormalTok{reduce\_homo }\OtherTok{\textless{}{-}} \FunctionTok{feols}\NormalTok{(educ }\SpecialCharTok{\textasciitilde{}}\NormalTok{ nearc4 }\SpecialCharTok{+}\NormalTok{ exper }\SpecialCharTok{+}\NormalTok{ expersq }\SpecialCharTok{+}\NormalTok{ black }\SpecialCharTok{+}\NormalTok{ south }\SpecialCharTok{+}\NormalTok{ smsa }\SpecialCharTok{+}\NormalTok{ smsa66 }\SpecialCharTok{+}
\NormalTok{                     reg661 }\SpecialCharTok{+}\NormalTok{ reg662 }\SpecialCharTok{+}\NormalTok{ reg663 }\SpecialCharTok{+}\NormalTok{ reg664 }\SpecialCharTok{+}\NormalTok{ reg665 }\SpecialCharTok{+}\NormalTok{ reg666 }\SpecialCharTok{+}\NormalTok{ reg667 }\SpecialCharTok{+}\NormalTok{ reg668,}
                     \AttributeTok{se =} \StringTok{"iid"}\NormalTok{, df)}

\NormalTok{reduce\_hetero }\OtherTok{\textless{}{-}} \FunctionTok{feols}\NormalTok{(educ }\SpecialCharTok{\textasciitilde{}}\NormalTok{ nearc4 }\SpecialCharTok{+}\NormalTok{ exper }\SpecialCharTok{+}\NormalTok{ expersq }\SpecialCharTok{+}\NormalTok{ black }\SpecialCharTok{+}\NormalTok{ south }\SpecialCharTok{+}\NormalTok{ smsa }\SpecialCharTok{+}\NormalTok{ smsa66 }\SpecialCharTok{+}
\NormalTok{                      reg661 }\SpecialCharTok{+}\NormalTok{ reg662 }\SpecialCharTok{+}\NormalTok{ reg663 }\SpecialCharTok{+}\NormalTok{ reg664 }\SpecialCharTok{+}\NormalTok{ reg665 }\SpecialCharTok{+}\NormalTok{ reg666 }\SpecialCharTok{+}\NormalTok{ reg667 }\SpecialCharTok{+}\NormalTok{ reg668,}
                      \AttributeTok{se =} \StringTok{"hc1"}\NormalTok{, df)}

\NormalTok{out }\OtherTok{\textless{}{-}} \FunctionTok{etable}\NormalTok{(reduce\_homo, reduce\_hetero, }\AttributeTok{tex =} \ConstantTok{TRUE}\NormalTok{, }\AttributeTok{se.row =} \ConstantTok{TRUE}\NormalTok{) }
\end{Highlighting}
\end{Shaded}

\begin{Shaded}
\begin{Highlighting}[]
\NormalTok{knitr}\SpecialCharTok{::}\FunctionTok{asis\_output}\NormalTok{(}\FunctionTok{c}\NormalTok{(}\StringTok{"}\SpecialCharTok{\textbackslash{}\textbackslash{}}\StringTok{begin\{center\}"}\NormalTok{, out, }\StringTok{"}\SpecialCharTok{\textbackslash{}\textbackslash{}}\StringTok{end\{center\}"}\NormalTok{)) }
\end{Highlighting}
\end{Shaded}

\begin{center}\begingroup\centering\begin{tabular}{lcc}   \tabularnewline \midrule \midrule   Dependent Variable: & \multicolumn{2}{c}{educ}\\   Model:          & (1)             & (2)\\     \midrule   \emph{Variables}\\   (Intercept)     & 16.85$^{***}$   & 16.85$^{***}$\\                      & (0.2111)        & (0.1866)\\      nearc4          & 0.3199$^{***}$  & 0.3199$^{***}$\\                      & (0.0879)        & (0.0851)\\      exper           & -0.4125$^{***}$ & -0.4125$^{***}$\\                      & (0.0337)        & (0.0321)\\      expersq         & 0.0009          & 0.0009\\                      & (0.0016)        & (0.0017)\\      black           & -0.9355$^{***}$ & -0.9355$^{***}$\\                      & (0.0937)        & (0.0925)\\      south           & -0.0516         & -0.0516\\                      & (0.1354)        & (0.1420)\\      smsa            & 0.4022$^{***}$  & 0.4022$^{***}$\\                      & (0.1048)        & (0.1112)\\      smsa66          & 0.0255          & 0.0255\\                      & (0.1058)        & (0.1106)\\      reg661          & -0.2103         & -0.2103\\                      & (0.2025)        & (0.1994)\\      reg662          & -0.2889$^{**}$  & -0.2889$^{*}$\\                      & (0.1473)        & (0.1513)\\      reg663          & -0.2382$^{*}$   & -0.2382$^{*}$\\                      & (0.1426)        & (0.1431)\\      reg664          & -0.0931         & -0.0931\\                      & (0.1860)        & (0.1799)\\      reg665          & -0.4829$^{**}$  & -0.4829$^{**}$\\                      & (0.1882)        & (0.1951)\\      reg666          & -0.5131$^{**}$  & -0.5131$^{**}$\\                      & (0.2096)        & (0.2090)\\      reg667          & -0.4271$^{**}$  & -0.4271$^{**}$\\                      & (0.2056)        & (0.2110)\\      reg668          & 0.3136          & 0.3136\\                      & (0.2417)        & (0.2338)\\      \midrule   \emph{Fit statistics}\\   Standard-Errors & IID             & Heteroskedasticity-robust \\      Observations    & 3,010           & 3,010\\     R$^2$           & 0.47712         & 0.47712\\     Adjusted R$^2$  & 0.47450         & 0.47450\\     \midrule \midrule   \multicolumn{3}{l}{\emph{Signif. Codes: ***: 0.01, **: 0.05, *: 0.1}}\\\end{tabular}\par\endgroup\end{center}

\newpage

\hypertarget{c-1}{%
\subsubsection{(c)}\label{c-1}}

The IV estimate has wider CI and the lower bound is closer to 0, hence
it's more conservative. However, the estimates from both IV and OLS are
significant on \(95 \%\) level since the lower bounds are both larger
than 0.

\begin{Shaded}
\begin{Highlighting}[]
\NormalTok{iv\_c }\OtherTok{\textless{}{-}} \FunctionTok{feols}\NormalTok{(lwage }\SpecialCharTok{\textasciitilde{}}\NormalTok{ exper }\SpecialCharTok{+}\NormalTok{ expersq }\SpecialCharTok{+}\NormalTok{ black }\SpecialCharTok{+}\NormalTok{ south }\SpecialCharTok{+}\NormalTok{ smsa }\SpecialCharTok{+}\NormalTok{ smsa66 }\SpecialCharTok{+}
\NormalTok{              reg661 }\SpecialCharTok{+}\NormalTok{ reg662 }\SpecialCharTok{+}\NormalTok{ reg663 }\SpecialCharTok{+}\NormalTok{ reg664 }\SpecialCharTok{+}\NormalTok{ reg665 }\SpecialCharTok{+}\NormalTok{ reg666 }\SpecialCharTok{+}\NormalTok{ reg667 }\SpecialCharTok{+}\NormalTok{ reg668 }\SpecialCharTok{|}
\NormalTok{              educ }\SpecialCharTok{\textasciitilde{}}\NormalTok{ nearc4,}
              \AttributeTok{se =} \StringTok{"hc1"}\NormalTok{, df)}

\NormalTok{out }\OtherTok{\textless{}{-}} \FunctionTok{etable}\NormalTok{(iv\_c, hetero, }\AttributeTok{tex =} \ConstantTok{TRUE}\NormalTok{, }
              \AttributeTok{coefstat =} \FunctionTok{c}\NormalTok{(}\StringTok{"confint"}\NormalTok{),}
              \AttributeTok{headers=}\FunctionTok{list}\NormalTok{(}\StringTok{"IV"} \OtherTok{=} \DecValTok{1}\NormalTok{, }\StringTok{"OLS"} \OtherTok{=} \DecValTok{1}\NormalTok{)) }
\end{Highlighting}
\end{Shaded}

\begin{Shaded}
\begin{Highlighting}[]
\NormalTok{knitr}\SpecialCharTok{::}\FunctionTok{asis\_output}\NormalTok{(}\FunctionTok{c}\NormalTok{(}\StringTok{"}\SpecialCharTok{\textbackslash{}\textbackslash{}}\StringTok{begin\{center\}"}\NormalTok{, out, }\StringTok{"}\SpecialCharTok{\textbackslash{}\textbackslash{}}\StringTok{end\{center\}"}\NormalTok{)) }
\end{Highlighting}
\end{Shaded}

\begin{center}\begingroup\centering\begin{tabular}{lcc}   \tabularnewline \midrule \midrule   Dependent Variable: & \multicolumn{2}{c}{lwage}\\                  & IV                 & OLS \\      Model:         & (1)                & (2)\\     \midrule   \emph{Variables}\\   (Intercept)    & 3.774$^{***}$      & 4.739$^{***}$\\                     & [1.970; 5.578]     & [4.593; 4.886]\\      educ           & 0.1315$^{**}$      & 0.0747$^{***}$\\                     & [0.0253; 0.2377]   & [0.0675; 0.0818]\\      exper          & 0.1083$^{***}$     & 0.0848$^{***}$\\                     & [0.0624; 0.1542]   & [0.0716; 0.0981]\\      expersq        & -0.0023$^{***}$    & -0.0023$^{***}$\\                     & [-0.0030; -0.0017] & [-0.0029; -0.0017]\\      black          & -0.1468$^{***}$    & -0.1990$^{***}$\\                     & [-0.2497; -0.0438] & [-0.2346; -0.1634]\\      south          & -0.1447$^{***}$    & -0.1480$^{***}$\\                     & [-0.2018; -0.0875] & [-0.2029; -0.0930]\\      smsa           & 0.1118$^{***}$     & 0.1364$^{***}$\\                     & [0.0507; 0.1729]   & [0.0987; 0.1741]\\      smsa66         & 0.0185             & 0.0262\\                     & [-0.0218; 0.0588]  & [-0.0102; 0.0627]\\      reg661         & -0.1078$^{***}$    & -0.1186$^{***}$\\                     & [-0.1884; -0.0273] & [-0.1946; -0.0425]\\      reg662         & -0.0070            & -0.0222\\                     & [-0.0733; 0.0592]  & [-0.0809; 0.0365]\\      reg663         & 0.0404             & 0.0260\\                     & [-0.0235; 0.1044]  & [-0.0299; 0.0818]\\      reg664         & -0.0579            & -0.0635$^{*}$\\                     & [-0.1350; 0.0192]  & [-0.1357; 0.0087]\\      reg665         & 0.0385             & 0.0095\\                     & [-0.0588; 0.1357]  & [-0.0664; 0.0853]\\      reg666         & 0.0551             & 0.0220\\                     & [-0.0474; 0.1576]  & [-0.0586; 0.1025]\\      reg667         & 0.0268             & -0.0006\\                     & [-0.0717; 0.1253]  & [-0.0820; 0.0808]\\      reg668         & -0.1909$^{***}$    & -0.1750$^{***}$\\                     & [-0.2905; -0.0912] & [-0.2671; -0.0829]\\      \midrule   \emph{Fit statistics}\\   Observations   & 3,010              & 3,010\\     R$^2$          & 0.23817            & 0.29984\\     Adjusted R$^2$ & 0.23435            & 0.29633\\     \midrule \midrule   \multicolumn{3}{l}{\emph{Heteroskedasticity-robust co-variance matrix, 95\% confidence intervals in brackets}}\\   \multicolumn{3}{l}{\emph{Signif. Codes: ***: 0.01, **: 0.05, *: 0.1}}\\\end{tabular}\par\endgroup\end{center}

\newpage

\hypertarget{d-1}{%
\subsubsection{(d)}\label{d-1}}

The table below presents estimation results from the reduced form. After
adding \(nearc2\), the coefficient of \(nearc4\) is even a littler
larger and the \(se\) is essentially the same. However, the coefficient
of \(nearc2\) is way smaller than \(nearc4\) and not significant.
Therefore, \(nearc4\) is more strongly related to \(educ\) than
\(nearc2\). After adding \(nearc2\), the adjusted R square also
increased a little - the independent variables can jointly explain more
variation of \(educ\).

\begin{Shaded}
\begin{Highlighting}[]
\NormalTok{reduce\_d }\OtherTok{\textless{}{-}} \FunctionTok{feols}\NormalTok{(educ }\SpecialCharTok{\textasciitilde{}}\NormalTok{ nearc2 }\SpecialCharTok{+}\NormalTok{ nearc4 }\SpecialCharTok{+}\NormalTok{ exper }\SpecialCharTok{+}\NormalTok{ expersq }\SpecialCharTok{+}\NormalTok{ black }\SpecialCharTok{+}\NormalTok{ south }\SpecialCharTok{+}\NormalTok{ smsa }\SpecialCharTok{+}\NormalTok{ smsa66 }\SpecialCharTok{+}
\NormalTok{                         reg661 }\SpecialCharTok{+}\NormalTok{ reg662 }\SpecialCharTok{+}\NormalTok{ reg663 }\SpecialCharTok{+}\NormalTok{ reg664 }\SpecialCharTok{+}\NormalTok{ reg665 }\SpecialCharTok{+}\NormalTok{ reg666 }\SpecialCharTok{+}\NormalTok{ reg667 }\SpecialCharTok{+}\NormalTok{ reg668,}
                         \AttributeTok{se =} \StringTok{"hc1"}\NormalTok{, df)}

\NormalTok{out }\OtherTok{\textless{}{-}} \FunctionTok{etable}\NormalTok{(reduce\_d, reduce\_hetero, }\AttributeTok{tex =} \ConstantTok{TRUE}\NormalTok{, }\AttributeTok{se.row =} \ConstantTok{TRUE}\NormalTok{) }
\end{Highlighting}
\end{Shaded}

\begin{Shaded}
\begin{Highlighting}[]
\NormalTok{knitr}\SpecialCharTok{::}\FunctionTok{asis\_output}\NormalTok{(}\FunctionTok{c}\NormalTok{(}\StringTok{"}\SpecialCharTok{\textbackslash{}\textbackslash{}}\StringTok{begin\{center\}"}\NormalTok{, out, }\StringTok{"}\SpecialCharTok{\textbackslash{}\textbackslash{}}\StringTok{end\{center\}"}\NormalTok{)) }
\end{Highlighting}
\end{Shaded}

\begin{center}\begingroup\centering\begin{tabular}{lcc}   \tabularnewline \midrule \midrule   Dependent Variable: & \multicolumn{2}{c}{educ}\\   Model:         & (1)                   & (2)\\     \midrule   \emph{Variables}\\   (Intercept)    & 16.77$^{***}$         & 16.85$^{***}$\\                     & (0.1940)              & (0.1866)\\      nearc2         & 0.1230                &   \\                     & (0.0776)              &   \\      nearc4         & 0.3206$^{***}$        & 0.3199$^{***}$\\                     & (0.0850)              & (0.0851)\\      exper          & -0.4123$^{***}$       & -0.4125$^{***}$\\                     & (0.0320)              & (0.0321)\\      expersq        & 0.0008                & 0.0009\\                     & (0.0017)              & (0.0017)\\      black          & -0.9452$^{***}$       & -0.9355$^{***}$\\                     & (0.0925)              & (0.0925)\\      south          & -0.0419               & -0.0516\\                     & (0.1417)              & (0.1420)\\      smsa           & 0.4014$^{***}$        & 0.4022$^{***}$\\                     & (0.1113)              & (0.1112)\\      smsa66         & $7.82\times 10^{-5}$  & 0.0255\\                     & (0.1118)              & (0.1106)\\      reg661         & -0.1688               & -0.2103\\                     & (0.2009)              & (0.1994)\\      reg662         & -0.2690$^{*}$         & -0.2889$^{*}$\\                     & (0.1524)              & (0.1513)\\      reg663         & -0.1902               & -0.2382$^{*}$\\                     & (0.1468)              & (0.1431)\\      reg664         & -0.0377               & -0.0931\\                     & (0.1828)              & (0.1799)\\      reg665         & -0.4371$^{**}$        & -0.4829$^{**}$\\                     & (0.1979)              & (0.1951)\\      reg666         & -0.5022$^{**}$        & -0.5131$^{**}$\\                     & (0.2095)              & (0.2090)\\      reg667         & -0.3775$^{*}$         & -0.4271$^{**}$\\                     & (0.2144)              & (0.2110)\\      reg668         & 0.3820                & 0.3136\\                     & (0.2381)              & (0.2338)\\      \midrule   \emph{Fit statistics}\\   Standard-Errors & \multicolumn{2}{c}{Heteroskedasticity-robust} \\    Observations   & 3,010                 & 3,010\\     R$^2$          & 0.47756               & 0.47712\\     Adjusted R$^2$ & 0.47476               & 0.47450\\     \midrule \midrule   \multicolumn{3}{l}{\emph{Heteroskedasticity-robust standard-errors in parentheses}}\\   \multicolumn{3}{l}{\emph{Signif. Codes: ***: 0.01, **: 0.05, *: 0.1}}\\\end{tabular}\par\endgroup\end{center}

\newpage

After using two IV, the coefficient increases to \(0.16\), larger than
former IV result (\(0.13\)) and the se is even smaller, so it does
indicates a stronger relationship. The estimate is larger and we are
more confident that it's significantly different from 0.

\begin{Shaded}
\begin{Highlighting}[]
\NormalTok{iv\_d }\OtherTok{\textless{}{-}} \FunctionTok{feols}\NormalTok{(lwage }\SpecialCharTok{\textasciitilde{}}\NormalTok{ exper }\SpecialCharTok{+}\NormalTok{ expersq }\SpecialCharTok{+}\NormalTok{ black }\SpecialCharTok{+}\NormalTok{ south }\SpecialCharTok{+}\NormalTok{ smsa }\SpecialCharTok{+}\NormalTok{ smsa66 }\SpecialCharTok{+}
\NormalTok{              reg661 }\SpecialCharTok{+}\NormalTok{ reg662 }\SpecialCharTok{+}\NormalTok{ reg663 }\SpecialCharTok{+}\NormalTok{ reg664 }\SpecialCharTok{+}\NormalTok{ reg665 }\SpecialCharTok{+}\NormalTok{ reg666 }\SpecialCharTok{+}\NormalTok{ reg667 }\SpecialCharTok{+}\NormalTok{ reg668 }\SpecialCharTok{|}
\NormalTok{              educ }\SpecialCharTok{\textasciitilde{}}\NormalTok{ nearc2 }\SpecialCharTok{+}\NormalTok{ nearc4,}
              \AttributeTok{se =} \StringTok{"hc1"}\NormalTok{, df)}

\NormalTok{out }\OtherTok{\textless{}{-}} \FunctionTok{etable}\NormalTok{(iv\_d, iv\_c, }\AttributeTok{tex =} \ConstantTok{TRUE}\NormalTok{, }
              \AttributeTok{headers=}\FunctionTok{list}\NormalTok{(}\StringTok{"\#IV:2"} \OtherTok{=} \DecValTok{1}\NormalTok{, }\StringTok{"\#IV:1"} \OtherTok{=} \DecValTok{1}\NormalTok{)) }
\end{Highlighting}
\end{Shaded}

\begin{Shaded}
\begin{Highlighting}[]
\NormalTok{knitr}\SpecialCharTok{::}\FunctionTok{asis\_output}\NormalTok{(}\FunctionTok{c}\NormalTok{(}\StringTok{"}\SpecialCharTok{\textbackslash{}\textbackslash{}}\StringTok{begin\{center\}"}\NormalTok{, out, }\StringTok{"}\SpecialCharTok{\textbackslash{}\textbackslash{}}\StringTok{end\{center\}"}\NormalTok{)) }
\end{Highlighting}
\end{Shaded}

\begin{center}\begingroup\centering\begin{tabular}{lcc}   \tabularnewline \midrule \midrule   Dependent Variable: & \multicolumn{2}{c}{lwage}\\                  & \#IV:2          & \#IV:1 \\       Model:         & (1)             & (2)\\     \midrule   \emph{Variables}\\   (Intercept)    & 3.340$^{***}$   & 3.774$^{***}$\\                     & (0.8933)        & (0.9199)\\      educ           & 0.1571$^{***}$  & 0.1315$^{**}$\\                     & (0.0525)        & (0.0541)\\      exper          & 0.1188$^{***}$  & 0.1083$^{***}$\\                     & (0.0230)        & (0.0234)\\      expersq        & -0.0024$^{***}$ & -0.0023$^{***}$\\                     & (0.0004)        & (0.0003)\\      black          & -0.1233$^{**}$  & -0.1468$^{***}$\\                     & (0.0516)        & (0.0525)\\      south          & -0.1432$^{***}$ & -0.1447$^{***}$\\                     & (0.0303)        & (0.0291)\\      smsa           & 0.1008$^{***}$  & 0.1118$^{***}$\\                     & (0.0314)        & (0.0311)\\      smsa66         & 0.0151          & 0.0185\\                     & (0.0212)        & (0.0206)\\      reg661         & -0.1030$^{**}$  & -0.1078$^{***}$\\                     & (0.0427)        & (0.0411)\\      reg662         & -0.0002         & -0.0070\\                     & (0.0346)        & (0.0338)\\      reg663         & 0.0470          & 0.0404\\                     & (0.0336)        & (0.0326)\\      reg664         & -0.0554         & -0.0579\\                     & (0.0410)        & (0.0393)\\      reg665         & 0.0515          & 0.0385\\                     & (0.0508)        & (0.0496)\\      reg666         & 0.0700          & 0.0551\\                     & (0.0536)        & (0.0523)\\      reg667         & 0.0391          & 0.0268\\                     & (0.0516)        & (0.0502)\\      reg668         & -0.1980$^{***}$ & -0.1909$^{***}$\\                     & (0.0524)        & (0.0508)\\      \midrule   \emph{Fit statistics}\\   Observations   & 3,010           & 3,010\\     R$^2$          & 0.17020         & 0.23817\\     Adjusted R$^2$ & 0.16605         & 0.23435\\     \midrule \midrule   \multicolumn{3}{l}{\emph{Heteroskedasticity-robust standard-errors in parentheses}}\\   \multicolumn{3}{l}{\emph{Signif. Codes: ***: 0.01, **: 0.05, *: 0.1}}\\\end{tabular}\par\endgroup\end{center}

\newpage

\hypertarget{e}{%
\subsubsection{(e)}\label{e}}

\(IQ\) is significantly correlated with \(nearc4\). Intuitively, \(IQ\)
is also correlated with \(education\), so \(near4\) might not be a valid
IV for \(education\) since it does not satisfy the exclusion assumption.
The previous IV estimations might be biased.

\begin{Shaded}
\begin{Highlighting}[]
\NormalTok{df}\SpecialCharTok{$}\NormalTok{IQ }\OtherTok{\textless{}{-}} \FunctionTok{as.numeric}\NormalTok{(df}\SpecialCharTok{$}\NormalTok{IQ)}
\NormalTok{ols\_e }\OtherTok{\textless{}{-}} \FunctionTok{feols}\NormalTok{(IQ }\SpecialCharTok{\textasciitilde{}}\NormalTok{ nearc4, }\AttributeTok{se =} \StringTok{"hc1"}\NormalTok{, df)}
\NormalTok{out }\OtherTok{\textless{}{-}} \FunctionTok{etable}\NormalTok{(ols\_e, }\AttributeTok{tex =} \ConstantTok{TRUE}\NormalTok{) }
\end{Highlighting}
\end{Shaded}

\begin{Shaded}
\begin{Highlighting}[]
\NormalTok{knitr}\SpecialCharTok{::}\FunctionTok{asis\_output}\NormalTok{(}\FunctionTok{c}\NormalTok{(}\StringTok{"}\SpecialCharTok{\textbackslash{}\textbackslash{}}\StringTok{begin\{center\}"}\NormalTok{, out, }\StringTok{"}\SpecialCharTok{\textbackslash{}\textbackslash{}}\StringTok{end\{center\}"}\NormalTok{)) }
\end{Highlighting}
\end{Shaded}

\begin{center}\begingroup\centering\begin{tabular}{lc}   \tabularnewline \midrule \midrule   Dependent Variable: & IQ\\     Model:              & (1)\\     \midrule   \emph{Variables}\\   (Intercept)         & 100.6$^{***}$\\                          & (0.6331)\\      nearc4              & 2.596$^{***}$\\                          & (0.7495)\\      \midrule   \emph{Fit statistics}\\   Observations        & 2,061\\     R$^2$               & 0.00586\\     Adjusted R$^2$      & 0.00537\\     \midrule \midrule   \multicolumn{2}{l}{\emph{Heteroskedasticity-robust standard-errors in parentheses}}\\   \multicolumn{2}{l}{\emph{Signif. Codes: ***: 0.01, **: 0.05, *: 0.1}}\\\end{tabular}\par\endgroup\end{center}

\newpage

\hypertarget{f}{%
\subsubsection{(f)}\label{f}}

After controlling for \(region\) variables, the coefficient of
\(nearc4\) is not significant anymore. Therefore, after controlling for
these \(region\) variables, \(nearc4\) can serve as a valid IV for
\(education\) since we can exclude the bias led by \(IQ\).

\begin{Shaded}
\begin{Highlighting}[]
\NormalTok{df}\SpecialCharTok{$}\NormalTok{IQ }\OtherTok{\textless{}{-}} \FunctionTok{as.numeric}\NormalTok{(df}\SpecialCharTok{$}\NormalTok{IQ)}
\NormalTok{ols\_f }\OtherTok{\textless{}{-}} \FunctionTok{feols}\NormalTok{(IQ }\SpecialCharTok{\textasciitilde{}}\NormalTok{ nearc4 }\SpecialCharTok{+}\NormalTok{ smsa66 }\SpecialCharTok{+}\NormalTok{ reg661 }\SpecialCharTok{+}\NormalTok{ reg662 }\SpecialCharTok{+}\NormalTok{ reg669, }\AttributeTok{se =} \StringTok{"hc1"}\NormalTok{, df)}
\NormalTok{out }\OtherTok{\textless{}{-}} \FunctionTok{etable}\NormalTok{(ols\_f, ols\_e, }\AttributeTok{tex =} \ConstantTok{TRUE}\NormalTok{) }
\end{Highlighting}
\end{Shaded}

\begin{Shaded}
\begin{Highlighting}[]
\NormalTok{knitr}\SpecialCharTok{::}\FunctionTok{asis\_output}\NormalTok{(}\FunctionTok{c}\NormalTok{(}\StringTok{"}\SpecialCharTok{\textbackslash{}\textbackslash{}}\StringTok{begin\{center\}"}\NormalTok{, out, }\StringTok{"}\SpecialCharTok{\textbackslash{}\textbackslash{}}\StringTok{end\{center\}"}\NormalTok{)) }
\end{Highlighting}
\end{Shaded}

\begin{center}\begingroup\centering\begin{tabular}{lcc}   \tabularnewline \midrule \midrule   Dependent Variable: & \multicolumn{2}{c}{IQ}\\   Model:         & (1)           & (2)\\     \midrule   \emph{Variables}\\   (Intercept)    & 99.38$^{***}$ & 100.6$^{***}$\\                     & (0.7135)      & (0.6331)\\      nearc4         & 0.8681        & 2.596$^{***}$\\                     & (0.8183)      & (0.7495)\\      smsa66         & 1.355$^{*}$   &   \\                     & (0.7904)      &   \\      reg661         & 4.768$^{***}$ &   \\                     & (1.423)       &   \\      reg662         & 5.808$^{***}$ &   \\                     & (0.8679)      &   \\      reg669         & 1.845         &   \\                     & (1.142)       &   \\      \midrule   \emph{Fit statistics}\\   Observations   & 2,061         & 2,061\\     R$^2$          & 0.03018       & 0.00586\\     Adjusted R$^2$ & 0.02783       & 0.00537\\     \midrule \midrule   \multicolumn{3}{l}{\emph{Heteroskedasticity-robust standard-errors in parentheses}}\\   \multicolumn{3}{l}{\emph{Signif. Codes: ***: 0.01, **: 0.05, *: 0.1}}\\\end{tabular}\par\endgroup\end{center}

\end{document}
